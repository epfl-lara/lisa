\chapter{Developping Mathematics with Prooflib}
\label{chapt:prooflib}
LISA's kernel offers all the necessary tools to develops proofs, but  reading and writing proofs written directly in its language is cumbersome. 
To develop and maintain a library of mathematical development, LISA offers a dedicate interface and DSL to write proofs: Prooflib
LISA provides a canonical way of writing and organizing Kernel proofs by mean of a set of utilities and a DSL made possible by some of Scala 3's features.
\autoref{fig:theoryFileExample} is a reminder from \autoref{chapt:quickguide} of the canonical way to write a theory file in LISA.

\begin{figure}
\begin{lstlisting}[language=lisa, frame=single]
object MyTheoryName extends lisa.Main {
  val x = variable
  val f = function[1]
  val P = predicate[1]

  val fixedPointDoubleApplication = Theorem( 
    ∀(x, P(x) ==> P(f(x))) |- P(x) ==> P(f(f(x)))
  ) {
    assume(∀(x, P(x) ==> P(f(x))))
    val step1 = have(P(x) ==> P(f(x))) by InstantiateForall
    val step2 = have(P(f(x)) ==> P(f(f(x)))) by InstantiateForall
    have(thesis) by Tautology.from(step1, step2)
  } 


  val emptySetIsASubset = Theorem(
    ∅ ⊆ x
  ) {
    have((y ∈ ∅) ==> (y ∈ x)) by Tautology.from(
                            emptySetAxiom of (x := y))
    val rhs = thenHave (∀(y, (y ∈ ∅) ==> (y ∈ x))) by RightForall
    have(thesis) by Tautology.from(
                            subsetAxiom of (x := ∅, y := x), rhs)
  }

}
\end{lstlisting}
\caption{An example of a theory file in LISA}
\label{fig:theoryFileExample}
\end{figure}

In this chapter, we will describe how each of these construct is made possible and how they translate to statements in the Kernel.

\section*{WIP}

%It is important to note that when multiple such files are developed, they all use the same underlying \lstinline|RunningTheory|{}. This makes it possible to use results proved previously by means of a simple \lstinline|import|{} statement as one would import a regular object. Similarly, one should also import as usual automation and tactics developed alongside. It is expected in the medium term that \lstinline|lisa.Main|{} will come with basic automation.
%All imported theory objects will be run through as well, but only the result of the selected one will be printed.





