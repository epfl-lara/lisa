%\part{Theory}

\chapter{Library Development: Set Theory}
\label{chapt:settheory}

It is important to remember that in the context of Set Theory, function symbols are not the usual mathematical functions and predicate symbols are not the usual mathematical predicates. Indeed, a predicate on the natural numbers $\mathbb N$ is simply a subset of $\mathbb N$. For example a number is even if and only if it is in the set $E \subset \mathbb N$ of all even numbers. Similarly, the $\leq$ relation on natural numbers can be thought of as a subset of $\mathbb N \times \mathbb N$. There, $E$ and $\leq$ are themselves sets, and in particular terms in first order logic.
Actual mathematical functions on the other hand, are proper sets which contains the graph of a function on some domain. Their domain must be restricted to a proper set, and it is possible to quantify over such set-like functions or to use them without applications. These set-like functions are represented by constant symbols.  For example ``$f$ is derivable'' cannot be stated about a function symbol. We will come back to this in Chapter~\ref{chapt:settheory}, but for now let us remember that (non-constant) function symbols are suitable for intersection ($\bigcap$) between sets but not for, say, the Riemann $\zeta$ function.


Indeed, on one hand a predicate symbol defines a truth value on all possible sets, but on the other hand it is impossible to use the symbol alone, without applying it to arguments, or to quantify over function symbol.

LISA is based on set theory. More specifically, it is based on ZF with (still not decided) an axiom of choice, of global choice, or Tarski's universes.

ZF Set Theory stands for Zermelo-Fraenkel Set Theory. It contains a set of initial predicate symbols and function symbols, as shown in Figure \ref{fig:symbolszf}. It also contains the 7 axioms of Zermelo (Figure \ref{fig:axiomsz}), which are technically sufficient to formalize a large portion of mathematics, plus the axiom of replacement of Fraenkel (Figure \ref{fig:axiomszf}), which is needed to formalize more complex mathematical theories.
In a more typical mathematical introduction to Set Theory, ZF would naturally only contain the set membership symbol $\in$. Axioms defining the other symbols would then only express the existence of functions or predicates with those properties, from which we could get the same symbols using extensions by definitions.

In a very traditional sense, an axiomatization is any possibly infinite semi-recursive set of axioms. Hence, in its full generality, Axioms should be any function producing possibly infinitely many formulas.
This is however not a convenient definition. In practice, all infinite axiomatizations are schematic, meaning that they are expressable using schematic variables. Axioms \ref{axz:comprehension} (comprehension schema) and \ref{axzf:replacement} (replacement schema) are such examples of axiom schema, and motivates the use of schematic variables in LISA.



\begin{figure}
  \begin{center}
    \begin{tabular}{l|c|l}
      {}                         & Math symbol       & LISA Kernel             \\ \hline
      Set Membership predicate   & $\in$             & \lstinline$in(s,t)$     \\
      Subset predicate           & $\subset$         & \lstinline$subset(s,t)$ \\
      Empty Set constant         & $\emptyset$       & \lstinline$emptyset()$  \\
      Unordered Pair constant    & $(\cdot, \cdot )$ & \lstinline$pair(s,t)$   \\
      Power Set function         & $\mathcal P$      & \lstinline$powerSet(s)$ \\
      Set Union/Flatten function & $\bigcup$         & \lstinline$union(x)$    \\
    \end{tabular}

    \caption{The basic symbols of ZF.}
    \label{fig:symbolszf}
  \end{center}
\end{figure}

\begin{figure}
  \begin{axz}[empty set]\label{axz:empty}
    $\forall x. x \notin \emptyset$
  \end{axz}
  \begin{axz}[extensionality]\label{axz:extensionality}
    $\forall x, y. (\forall z. z \in x \iff z \in y) \iff (x = y)$
  \end{axz}
  \begin{axz}[subset]\label{axz:subset}
    $\forall x, y. x\subset y \iff \forall z. z \in x \iff z \in y$
  \end{axz}
  \begin{axz}[pair]\label{axz:pair}
    $\forall x, y, z. (z \in (x, y)) \iff ((x \in y) \lor (y \in z))$
  \end{axz}
  \begin{axz}[union]\label{axz:union}
    $\forall x, z. (x \in \operatorname{U}(z)) \iff (\exists y. (x \in y) \land (y \in z))$
  \end{axz}
  \begin{axz}[power]\label{axz:power}
    $\forall x, y. (x \in \operatorname{\mathcal{P}}(y)) \iff (x \subset y)$
  \end{axz}
  \begin{axz}[foundation]\label{axz:foundation}
    $\forall x. (x \neq \emptyset) \implies (\exists y. (y \in x) \land (\forall z. z \in x))$
  \end{axz}
  \begin{axz}[comprehension schema]\label{axz:comprehension}
    $\forall z, \vec{v}. \exists y. \forall x. (x \in y) \iff ((x \in z) \land \phi(x, z, \vec{v}))$
  \end{axz}
  \caption{Axioms for Zermelo set theory.}
  \label{fig:axiomsz}
\end{figure}

\begin{figure}
  \begin{axzf}[replacement schema]\label{axzf:replacement}
    $$\forall a. (\forall x. (x \in a) \implies !\exists y. \phi(a, \vec{v}, x, y)) \implies $$
    $$(\exists b. \forall x. (x \in a) \implies (\exists y. (y \in b) \land \phi(a, \vec{v}, x, y)))$$
  \end{axzf}
  \caption{Axioms for Zermelo-Fraenkel set theory.}
  \label{fig:axiomszf}
\end{figure}
