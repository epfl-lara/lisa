% macro.tex

%\usepackage[utf8]{inputenc}
\usepackage{url}
\usepackage{graphicx}
\usepackage[english]{babel}
\usepackage{amsmath}
\usepackage{amsfonts}
\usepackage{amssymb}
\usepackage{amsthm}
\usepackage{listings}
\usepackage[dvipsnames]{xcolor}
\usepackage{csquotes}
\usepackage[strings]{underscore}
\usepackage{bussproofs}
\usepackage{makecell}
\usepackage{subcaption}
\usepackage[dvipsnames]{xcolor}
\usepackage{float}
\usepackage{xspace}
\usepackage[ruled,vlined]{algorithm2e}
\usepackage{array}
\usepackage[T1]{fontenc}
\usepackage{newunicodechar} %To map unicode in listings to a different font than Fira Code
\usepackage{tabularx}
\renewcommand\sfdefault{ua1}

\sloppy % better (?) margin handling


\newtheorem{theorem}{Theorem}
\newtheorem{lemma}{Lemma}
\newtheorem{corollary}{Corollary}

\newtheorem{axz}{}
\renewcommand\theaxz{Z\arabic{axz}}
\newcommand*{\axzautorefname}{Axiom}
\newtheorem{axzf}{}
\renewcommand\theaxzf{ZF\arabic{axzf}}
\newcommand*{\axzfautorefname}{Axiom}
\newtheorem{axtg}{}
\renewcommand\theaxtg{TG\arabic{axtg}}
\newcommand*{\axtgautorefname}{Axiom}

\theoremstyle{definition}
\newtheorem{definition}{Definition}
\newtheorem{example}{Example}
\newcommand*{\definitionautorefname}{Definition}


\newcommand*{\listfigurenameautorefname}{Listing}


% Shortcuts
\newcommand{\Prop}{\text{Prop}}
\newcommand{\Ind}{\text{Ind}}
\newcommand{\lambdafol}{$\lambda$-FOL}




%  Code fonts

%\usepackage{lstfiracode}
\usepackage{fontspec}
\usepackage{ucharclasses}
\setmonofont[
  Contextuals={Alternate}
]{Fira Code}
\newfontfamily{\fallbackfont}{DejaVuSans}
\DeclareTextFontCommand{\textfallback}{\fallbackfont}
\newunicodechar{∀}{\textfallback{∀}}
\newunicodechar{∃}{\textfallback{∃}}
\newunicodechar{∅}{\textfallback{∅}}
\newunicodechar{⊆}{\textfallback{⊆}}
\newunicodechar{∈}{\textfallback{∈}}


%  Correct lstlisting parsing of unicode character. Add unicode points at the end.
% To add: code for \vdash 
\makeatletter
\lst@InputCatcodes
\def\lst@DefEC{%
 \lst@CCECUse \lst@ProcessLetter
  ^^80^^81^^82^^83^^84^^85^^86^^87^^88^^89^^8a^^8b^^8c^^8d^^8e^^8f%
  ^^90^^91^^92^^93^^94^^95^^96^^97^^98^^99^^9a^^9b^^9c^^9d^^9e^^9f%
  ^^a0^^a1^^a2^^a3^^a4^^a5^^a6^^a7^^a8^^a9^^aa^^ab^^ac^^ad^^ae^^af%
  ^^b0^^b1^^b2^^b3^^b4^^b5^^b6^^b7^^b8^^b9^^ba^^bb^^bc^^bd^^be^^bf%
  ^^c0^^c1^^c2^^c3^^c4^^c5^^c6^^c7^^c8^^c9^^ca^^cb^^cc^^cd^^ce^^cf%
  ^^d0^^d1^^d2^^d3^^d4^^d5^^d6^^d7^^d8^^d9^^da^^db^^dc^^dd^^de^^df%
  ^^e0^^e1^^e2^^e3^^e4^^e5^^e6^^e7^^e8^^e9^^ea^^eb^^ec^^ed^^ee^^ef%
  ^^f0^^f1^^f2^^f3^^f4^^f5^^f6^^f7^^f8^^f9^^fa^^fb^^fc^^fd^^fe^^ff%
  ^^^^2200^^^^2203^^^^2205^^^^2286^^^^2208% <--- for ∀
  ^^00}
\lst@RestoreCatcodes
\makeatother


%  Syntactic coloring


\makeatletter
\def\verbatim@nolig@list{}
\makeatother

\definecolor{green}{rgb}{0, 0.35, 0}

\definecolor{comments}{RGB}{80,0,110}

\lstdefinelanguage{scala}{
    keepspaces=true,
    extendedchars = true,
    inputencoding = utf8,
    alsoletter={@,=,>},
    backgroundcolor = \color{rosishlightgray},
    keywordstyle = {\color{blue}},
    keywordstyle = [2]{\color{blue}},
    commentstyle = \color{comments},
    basicstyle=\footnotesize\linespread{1.15}\ttfamily,
    morekeywords = [2]{abstract, case, class, def, do, Input, Output, then,
        else, extends, false, free, if, implicit, match,
        object, true, val, var, while, sealed, or,
        for, dependent, null, type, with, try, catch, finally,
        import, final, return, new, override, this, trait,
        private, public, protected, package, throw},
    sensitive = true, 
    %numbers=left,
    stepnumber=1,
    morecomment = [l]{//},
    morecomment = [s]{/*}{*/},
    morestring = [b]",  
    otherkeywords = {;,<<,>>,++},
}


\lstdefinelanguage{lisa}{
    keepspaces=true,
    extendedchars = true,
    inputencoding = utf8,
    alsoletter={@,=,>},
    backgroundcolor = \color{rosishlightgray},
    keywordstyle = {\color{blue}},
    keywordstyle = [2]{\color{blue}},
    keywordstyle = [3]{\color{green}},
    keywordstyle = [4]{\color{teal}},
    commentstyle = \color{comments},
    basicstyle=\footnotesize\linespread{1.15}\ttfamily,
    morekeywords = [2]{abstract, case, class, def, do, Input, Output, then,
        else, extends, false, free, if, implicit, match,
        object, true, val, var, while, sealed, or,
        for, dependent, null, type, with, try, catch, finally,
        import, final, return, new, override, this, trait,
        private, public, protected, package, throw, given},
    morekeywords = [3]{have, andThen, thenHave, Theorem, by, DEF, The, Lemma, subproof, assume},
    sensitive = true, 
    numbers=left,
    stepnumber=1,
    morecomment = [l]{//},
    morecomment = [s]{/*}{*/},
    morestring = [b]",  
    otherkeywords = {;,<<,>>,++},
}


\definecolor{rosishlightgray}{rgb}{0.96, 0.94, 0.92}
\definecolor{bluishlightgray}{rgb}{0.94, 0.96, 0.98}

\lstdefinelanguage{console}{
    keepspaces=true,
    extendedchars = true,
    inputencoding = utf8,
    backgroundcolor = \color{bluishlightgray},
    basicstyle=\footnotesize\linespread{1.15}\ttfamily,
    mathescape,
    escapeinside={!*}{*!},
    columns=fullflexible,
    morekeywords={if,then,else,return,match,with,end,let,in, data, type, :=, def},
    moredelim=**[is][\color{green}]{@*}{*@},
    moredelim=**[is][\color{red}]{@**}{**@},
}

\lstset{
    keepspaces=true,
    extendedchars = true,
    inputencoding = utf8,
    extendedchars = \true,
    basicstyle=\footnotesize\linespread{1.15}\ttfamily,
    mathescape,
    escapeinside={!*}{*!},
    columns=fullflexible,
    morekeywords={if,then,else,return,match,with,end,let,in, data, type, :=, def},
    moredelim=**[is][\color{green}]{@*}{*@},
    moredelim=**[is][\color{red}]{@**}{**@},
}

\DeclareMathOperator{\pick}{pick}

\newcommand{\lisaCode}[3]{
\lstinputlisting[language=lisa, frame=single,caption=\href{#1}{#2}, #3]{#1}
}
