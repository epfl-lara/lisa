\chapter{Starting with Proofs in Lisa}
\label{chapt:quickguide}
Lisa is a proof assistant. Proof assistants support development of formal proofs of mathematical statements. 

The centerpiece of Lisa (called the \emph{kernel}) contains a mechanized implementation of first order logic (FOL), a logical framework to write mathematical statements and their proofs. This kernel is what provides correctness guarantees to the user. The kernel only accepts a small set of formal deduction rule such as ``if $a$ is true and $b$ is true then $a\land b$ is true''.
This is in contrast to human-written proofs, which may contain a wide variety of complex or implicit arguments. If a proof is accepted as being correct by the kernel, it is expected to meet objective criteria for valid proofs according to the field of formal mathematical logic.\footnote{It is possible that the kernel itself has an implementation bug, but because it is a very small and simple program available in open source, we can build strong confidence that it is correct.}
Lisa's kernel is described in more detail in \autoref{chapt:kernel}.

Writing mathematical theories (for example, group theory, combinatorics, topology, theory of computation) directly from these primitive constructions would be tedious. Instead, we use them as building blocs that can be combined and automatized. Beyond the correctness guarantees of the kernel, Lisa's purpose is to provide tools to make writing formal proofs easier. This include automation, via search procedures that automatically prove theorems, and layers of abstraction (helpers, domain specific language), which make the presentation of formal statements and proofs closer to the traditional, human way of writing proofs. 
This is similar to programming languages: machine language is sufficient to write any program on a computer, but high level programming languages offer many convenient features which make writing complex programs easier and which are ultimately translated into assembly automatically. 
\autoref{chapt:prooflib} explains how these layers of abstraction and automation work. The rest of the present chapter gives a quick guide on how to use Lisa.

\section{Installation}
Lisa requires the Scala programming language to run. You can download and install Scala following the instructions at the Scala home page\footnote{\url{www.scala-lang.org/}}. Subsequently, clone the Lisa git repository:
\begin{lstlisting}[language=console]
> git clone https://github.com/epfl-lara/lisa
\end{lstlisting}
To test your installation, do
\begin{lstlisting}[language=console]
> cd lisa
> sbt
\end{lstlisting}
\lstinline|sbt| is a tool to run a Scala project and manage versions and dependencies. Once inside sbt, run the following commands:
\begin{lstlisting}[language=console]
> project lisa-examples
> run
\end{lstlisting}
Wait for the Lisa codebase to be compiled and then press the number corresponding to "Example". You should obtain the result demonstrating some example theorems proven, such as the following:

\noindent\begin{minipage}{\linewidth}\vspace{1em}
\begin{lstlisting}[language=console]
  @*Theorem fixedPointDoubleApplication := 
    ∀'x. 'P('x) ==> 'P('f('x)) |- 'P('x) ==> 'P('f('f('x)))

  Theorem emptySetIsASubset := |- subsetOf(emptySet, 'x)

  Theorem setWithElementNonEmpty := 
    elem('y, 'x) |- ¬('x = emptySet)

  Theorem powerSetNonEmpty := |- ¬(powerSet('x) = emptySet)
  *@
\end{lstlisting}
\end{minipage}

\section{Development Environment}
To write Lisa proofs, you can use any text editor or IDE. We recommend using \emph{Visual Studio Code} (henceforth VSCode) with the \emph{Metals} plugin.

\subsection*{A Note on Special Characters}
Math often uses symbols beyond the Latin alphabet. Lisa usually admits both an English alphabet name and a unicode name for such symbols. By enabling \emph{font ligatures}, common character sequences, such as \lstinline|=|\lstinline|=|\lstinline|>| are rendered as \lstinline|==>|. 
The present document also uses the font \href{https://github.com/tonsky/FiraCode}{Fira Code}. Once installed on your system, you can activate it and ligatures on VSCode the following way:
\begin{enumerate}
  \item Press Ctrl-Shift-P
  \item Search for ``Open User Settings (JSON)''
  \item in the \lstinline|settings.json| file, add:
  \begin{lstlisting}
"editor.fontFamily": "'Fira Code', Consolas, monospace",
"editor.fontLigatures": true,
  \end{lstlisting}
\end{enumerate}
Other symbols such as \lstinline|∀| are Unicode symbols, which can be entered via their unicode code, depending on your OS\footnote{alt+numpad on windows, Ctrl-Shift-U+code on Linux}, or by using an extension for VS Code such as \emph{Fast Unicode Math Characters}, \emph{Insert Unicode} or \emph{Unicode Latex}.
A cheat sheet of the most common symbols and how to input them is in \autoref{tab:Unicode}.
\begin{table}
  \center
  \begin{tabular}{c|c|c}
    Rendering         & Input            & Name     \\ \hline
    \lstinline| === | & ===              & equality \\ \hline
    \lstinline| \/  | & \textbackslash / & or       \\ \hline
    \lstinline| /\  | & /\textbackslash  & and      \\ \hline
    \lstinline| ==> | & ==>              & implies  \\ \hline
    \lstinline+ |-  + &  |-              & vdash    \\ \hline
    \lstinline| ∀   | & U+2200           & forall   \\ \hline
    \lstinline| ∃   | & U+2203           & exists   \\ \hline
    \lstinline| ∈   | & U+2208           & in       \\ \hline
    \lstinline| ⊆   | & U+2286           & subseteq \\ \hline
    \lstinline| ∅   | & U+2205           & emptyset \\ 
  \end{tabular}
  \caption{Frequently used Unicode symbols and ligatures.}
  \label{tab:Unicode}
\end{table}
Note that by default, Unicode characters may not be printed correctly on a Windows console. You will need to activate the corresponding charset and pick a font with support for unicode in your console's options, such as Consolas.

\section{Writing theory files}
Lisa provides a canonical way of writing and organizing kernel proofs by mean of a set of utilities and a domain-specific language (DSL) made possible by some of Scala's features.
To prove some theorems by yourself, start by creating a file named \lstinline|MyTheoryName.scala| right next to the Example.scala file\footnote{The relative path is lisa/lisa-examples/src/main/scala}.
Then simply write:

\noindent\begin{minipage}{\linewidth}\vspace{1em}
\begin{lstlisting}[language=lisa, frame=single]
object MyTheoryName extends lisa.Main {

}
\end{lstlisting}
\end{minipage}
and that's it! This will give you access to all the necessary Lisa features. Let see how one can use them to prove a theorem:
$$
  \forall x. P(x) \implies P(f(x)) \vdash P(x) \implies P(f(f(x)))
$$
To state the theorem, we first need to tell Lisa that $x$ is a variable, $f$ is a function symbol and $P$ a predicate symbol. 

\noindent\begin{minipage}{\linewidth}\vspace{1em}
\begin{lstlisting}[language=lisa, frame=single]
object MyTheoryName extends lisa.Main {
  val x = variable
  val f = function[1]
  val P = predicate[1]

}
\end{lstlisting}
\end{minipage}

where \lstinline|[1]| indicates that the symbol is of arity 1 (it takes a single argument). The symbols \lstinline|x, f, P| are scala identifiers that can be freely used in theorem statements and proofs, but they are also formal symbols of FOL in Lisa's kernel. 
We now can state our theorem:

\noindent\begin{minipage}{\linewidth}\vspace{1em}
\begin{lstlisting}[language=lisa, frame=single]
object MyTheoryName extends lisa.Main {
  val x = variable
  val f = function[1]
  val P = predicate[1]

  val fixedPointDoubleApplication = Theorem(
    ∀(x, P(x) ==> P(f(x))) |- P(x) ==> P(f(f(x)))
  ) {
    ???  // Proof
  } 
}
\end{lstlisting}
\end{minipage}
The theorem will automatically be named \lstinline|fixedPointDoubleApplication|, like the name of the identifier it is assigned to, and will be available to reuse in future proofs. The proof itself is built using a sequence of proof step, which will update the status of the ongoing proof.

\noindent\begin{minipage}{\linewidth}\vspace{1em}
\begin{lstlisting}[language=lisa, frame=single]
object MyTheoryName extends lisa.Main {
  val x = variable
  val f = function[1]
  val P = predicate[1]

  val fixedPointDoubleApplication = Theorem( 
    ∀(x, P(x) ==> P(f(x))) |- P(x) ==> P(f(f(x)))
  ) {
    assume(∀(x, P(x) ==> P(f(x))))
    val step1 = have(P(x) ==> P(f(x))) by InstantiateForall
    val step2 = have(P(f(x)) ==> P(f(f(x)))) by InstantiateForall
    have(thesis) by Tautology.from(step1, step2)
  } 
}
\end{lstlisting}
\end{minipage}
First, we use the \lstinline|assume| construct in line 6.
This tells to Lisa that the assumed formula is understood as being implicitly on the left hand side of every statement in the rest of the proof. 

Then, we need to instantiate the quantified formula twice using a specialized tactic. In lines 7 and 8, we use \lstinline|have| to state that a formula or sequent is true (given the assumption inside \lstinline|assume|), and that the proof of this is produced by the tactic \lstinline|InstantiateForall|.
We'll see more about the interface of a tactic later. To be able to reuse intermediate steps at any point later, we also assign the intermediates step to a variable.

Finally, the last line says that the conclusion of the theorem itself, \lstinline|thesis|, can be proven using the tactic \lstinline|Tautology| and the two intermediate steps we reached. \lstinline|Tautology| is a tactic that is able to do reasoning with propositional connectives. It implements a complete decision procedure for propositional logic that is described in \autoref{tact:Tautology}.

Lisa is based on set theory, so you can also use set-theoretic primitives such as in the following theorem.

\noindent\begin{minipage}{\linewidth}\vspace{1em}
  \begin{lstlisting}[language=lisa, frame=single]
val emptySetIsASubset = Theorem(
  ∅ ⊆ x
) {
  have((y ∈ ∅) ==> (y ∈ x)) by Tautology.from(
                          emptySetAxiom of (x := y))
  val rhs = thenHave (∀(y, (y ∈ ∅) ==> (y ∈ x))) by RightForall
  have(thesis) by Tautology.from(
                          subsetAxiom of (x := ∅, y := x), rhs)
}
  \end{lstlisting}
\end{minipage}
We see a number of new constructs in this example. \lstinline|RightForall| is another tactic (in fact it corresponds to a core deduction rules of the kernel) that introduces a quantifier around a formula, if the bound variable is not free somewhere else in the sequent.
We also see in line 6 another construct: \lstinline|thenHave|. It is similar to \lstinline|have|, but it will automatically pass the previous statement to the tactic. Formally,
\noindent\begin{minipage}{\linewidth}\vspace{1em}
  \begin{lstlisting}[language=lisa, frame=single]
    have(X) by Tactic1
    thenHave (Y) by Tactic2
  \end{lstlisting}
\end{minipage}
is equivalent to

\noindent\begin{minipage}{\linewidth}\vspace{1em}
  \begin{lstlisting}[language=lisa, frame=single]
    val s1 = have(X) by Tactic1
    have (Y) by Tactic2(s1)
  \end{lstlisting}
\end{minipage}
\lstinline|thenHave| allows us to not give a name to every step when we're doing linear reasoning. Finally, in lines 5 and 8, we see that tactic can refer not only to steps of the current proof, but also to previously proven theorems and axioms, such as \lstinline|emptySetAxiom|. The \lstinline|of| keyword indicates the axiom (or step) is instantiated in a particular way. For example:
\noindent\begin{minipage}{\linewidth}\vspace{1em}
  \begin{lstlisting}[language=lisa, frame=single]
    emptySetAxiom             // ==  !(x ∈ ∅)
    emptySetAxiom of (x := y) // ==  !(y ∈ ∅)
  \end{lstlisting}
\end{minipage}

Lisa also allows to introduce definitions. There are essentially two kind of definitions, \emph{aliases} and definition via \emph{unique existence}.
An alias defines a constant, a function or predicate as being equal (or equivalent) to a given formula or term. For example,

\noindent\begin{minipage}{\linewidth}\vspace{1em}
  \begin{lstlisting}[language=lisa, frame=single]
  val succ = DEF(x) --> union(unorderedPair(x, singleton(x)))
  \end{lstlisting}
\end{minipage}
defines the function symbol \lstinline|succ| as the function taking a single argument $x$ and mapping it to the element $\bigcup \lbrace x, \lbrace x \rbrace \rbrace$\footnote{This correspond to the traditional encoding of the successor function for natural numbers in set theory.}.

The second way of defining an object is more complicated and involve proving the existence and uniqueness of an object. This is detailed in \autoref{chapt:kernel}.

You can now try to run the theory file you just wrote and verify if you made a mistake. To do so again do \lstinline|> run| in the sbt console and select the number corresponding to your file. 
If all the output is green, perfect! If there is an error, it can be either a syntax error reported at compilation or an error in the proof. In both case, the error message can sometimes be cryptic, but it should at least consistently indicates which line of your file is incorrect.

Alternatively, if you are using IntelliJ or VS Code and Metals, you can run your theory file directly in your IDE by clicking either on the green arrow (IntelliJ) or on ``run" (VS Code) next to your main object.


\section{Common Tactics}
\subsubsection*{Restate}
Restate is a tactic that reasons modulo ortholattices, a subtheory of boolean algebra (see \cite{guilloudFormulaNormalizationsVerification2023} and \autoref{subsec:equivalencechecker}). Formally, it is very efficient and can prove a lot of simple propositional transformations, but not everything that is true in classical logic. In particular, it can't prove that $(a\land b) \lor (a \land c) \iff a \land (b \lor c)$ is true. It can however prove very limited facts involving equality and quantifiers. Usage:

\begin{lstlisting}[language=lisa]
  have(statement) by Restate
\end{lstlisting}
tries to justify \lstinline|statement| by showing it is equivalent to \lstinline|True|.

\begin{lstlisting}[language=lisa]
  have(statement) by Restate(premise)
\end{lstlisting}
tries to justify \lstinline|statement| by showing it is equivalent to the previously proven \lstinline|premise|.

\subsubsection*{Tautology}\label{tact:Tautology}
\lstinline|Tautology| is a propositional solver based upon restate, but complete. It is able to prove every formula inference that holds in classical propositional logic. However, in the worst case its complexity can be exponential in the size of the formula. Usage:

\begin{lstlisting}[language=lisa]
  have(statement) by Tautology
\end{lstlisting}
Constructs a proof of \lstinline|statement|, if the statement is true and a proof of it using only classical propositional reasoning exists.

\begin{lstlisting}[language=lisa]
  have(statement) by Tautology.from(premise1, premise2,...)
\end{lstlisting}
Construct a proof of \lstinline|statement| from the previously proven \lstinline|premise1|, \lstinline|premise2|,... using propositional reasoning.


\subsubsection*{RightForall, InstantiateForall}
\lstinline|RightForall| will generalize a statement by quantifying it over free variables. For example,
\begin{lstlisting}[language=lisa]
  have(P(x)) by ???
  thenHave(∀(x, P(x))) by RightForall 
\end{lstlisting}
Note that if the statement inside \lstinline|have| has more than one formula, $x$ cannot appear (it cannot be \emph{free}) in any formula other than $P(x)$. It can also not appear in any assumption.

\lstinline|InstantiateForall| does the opposite: given a universally quantified statement, it will specialize it. For example:
\begin{lstlisting}[language=lisa]
  have(∀(x, P(x))) by ???
  thenHave(P(t)) by InstantiateForall 
\end{lstlisting}
for any arbitrary term \lstinline|t|.

\subsubsection*{Substitution}
Substitutions allows reasoning by substituting equal terms and equivalent formulas. Usage:
\begin{lstlisting}[language=lisa]
  have(statement) by Substitution.ApplyRules(subst*)(premise)
\end{lstlisting}

\lstinline|subst*| is an arbitrary number of substitution. Each of those can be a previously proven fact (or theorem or axiom), or a formula. They must all be of the form \lstinline|s === t| or \lstinline|A <=> B|, otherwise the tactic will fail. The \lstinline|premise| is a previously proven fact. The tactic will try to show that \lstinline|statement| can be obtained from \lstinline|premise| by applying the substitutions from \lstinline|subst|. In its simplest form,
\begin{lstlisting}[language=lisa]
  val subst = have(s === t) by ???
  have(P(s)) by ???
  thenHave(P(t)) by Substitution.ApplyRules(subst)
\end{lstlisting}

Moreover, \lstinline|Substitution| is also able to  automatically unify and instantiate subst rules. For example

\begin{lstlisting}[language=lisa]
  val subst = have(g(x, y) === g(y, x)) by ???
  have(P(g(3, 8))) by ???
  thenHave(P(g(8, 3))) by Substitution.ApplyRules(subst)
\end{lstlisting}

If a \lstinline|subst| is a formula rather than a proven fact, then it should be an assumption in the resulting statement. Similarly, if one of the substitution has an assumption, it should be in the resulting statement. For example,

\begin{lstlisting}[language=lisa]
  val subst = have(A |- Q(s) <=> P(s)) by ???
  have(Q(s) /\ s===f(t)) by ???
  thenHave(A, f(t) === t |- P(s) /\ s===t) 
      .by Substitution.ApplyRules(subst, f(t) === t)
\end{lstlisting}


\subsection*{File Options}
Some options can be set at the start of a file, which will affect the behaviour of Lisa. These options are intended for use at a development stage.

\vspace{2em}
\begin{tabularx}{0.9\columnwidth}{|l|X|}
  \lstinline|draft()| & Theorems outside of the current file are assumed to be true and not checked for correctness. This can speed up repetitive runs during proof drafts.
  \\[4ex]
  \lstinline|withCache()| & Kernel proofs will be stored in binary files when theorems are constructed. On future runs with the option enabled, theorems will be constructed from the stored low level proofs. This skips running tactics to construct or search for proofs.\\
  
\end{tabularx}