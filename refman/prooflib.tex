\chapter{Developping Mathematics with Prooflib}
\label{chapt:prooflib}
Lisa's kernel offers all the necessary tools to develops proofs, but  reading and writing proofs written directly in its language is cumbersome. 
To develop and maintain a library of mathematical development, Lisa offers a dedicate interface and DSL to write proofs: Prooflib
Lisa provides a canonical way of writing and organizing Kernel proofs by mean of a set of utilities and a DSL made possible by some of Scala 3's features.
\autoref{fig:theoryFileExample} is a reminder from \autoref{chapt:quickguide} of the canonical way to write a theory file in Lisa.

\begin{figure}
\begin{lstlisting}[language=lisa, frame=single]
object MyTheoryName extends lisa.Main {
  val x = variable
  val f = function[1]
  val P = predicate[1]

  val fixedPointDoubleApplication = Theorem( 
    ∀(x, P(x) ==> P(f(x))) |- P(x) ==> P(f(f(x)))
  ) {
    assume(∀(x, P(x) ==> P(f(x))))
    val step1 = have(P(x) ==> P(f(x))) by InstantiateForall
    val step2 = have(P(f(x)) ==> P(f(f(x)))) by InstantiateForall
    have(thesis) by Tautology.from(step1, step2)
  } 

  val emptySetIsASubset = Theorem(
    ∅ ⊆ x
  ) {
    have((y ∈ ∅) ==> (y ∈ x)) by Tautology.from(
                            emptySetAxiom of (x := y))
    val rhs = thenHave (∀(y, (y ∈ ∅) ==> (y ∈ x))) by RightForall
    have(thesis) by Tautology.from(
                            subsetAxiom of (x := ∅, y := x), rhs)
  }

}
\end{lstlisting}
\caption{An example of a theory file in Lisa}
\label{fig:theoryFileExample}
\end{figure}

In this chapter, we will describe how each of these construct is made possible and how they translate to statements in the Kernel.

\section{Richer FOL}



\section{Proof Builders}

\subsection{Proofs}

\subsection{Facts}

\subsection{Instantiations}

\subsection{Local Definitions}
\label{sec:localDefinitions}
The following line of reasoning is standard in mathematical proofs. Suppose we have already proven the following fact:
$$\exists x. P(x)$$
And want to prove the property $\phi$.
A proof of $\phi$ using the previous theorem would naturally be obtained the following way:
\begin{quotation}
  Since we have proven $\exists x. P(x)$, let $c$ be an arbitrary value such that $P(c)$ holds.
  Hence we prove $\phi$, using the fact that $P(c)$: (...).
\end{quotation}
However, introducing a definition locally corresponding to a statement of the form
$$\exists x. P(x)$$
is not a built-in feature of first order logic.  This can however be simulated by introducing a fresh variable symbol $c$, that must stay fresh in the rest of the proof, and the assumption $P(c)$. The rest of the proof is then carried out under this assumption. When the proof is finished, the end statement should not contain $c$ free as it is a \textit{local} definition, and the assumption can be eliminated using the LeftExists and Cut rules. Such a $c$ is called a \textit{witness}. 
Formally, the proof in (...) is a proof of $P(c) \vdash \phi$. This can be transformed into a proof of $\phi$ by mean of the following steps:
\begin{center}
  \AxiomC{$P(c) \vdash \phi$}
  \UnaryInfC{$\exists x. P(x) \vdash  \phi$}
  \RightLabel{\text { LeftExists}}
  \AxiomC{$\exists x. P(x)$}
  \RightLabel{\text { Cut}}
  \BinaryInfC{$\phi$}
\end{center}
Not that for this step to be correct, $c$ must not be free in $\phi$. This correspond to the fact that $c$ is an arbitrary free symbol.

This simulation is provided by Lisa through the \lstinline|witness|{} method. It takes as argument a fact showing $\exists x. P(x)$, and introduce a new symbol with the desired property. For an example, see figure \ref{fig:localDefinitionExample}.

\begin{figure}
  \begin{lstlisting}[language=lisa, frame=single]
    val existentialAxiom = Axiom(exists(x, in(x, emptySet)))
    val falso = Theorem( ⊥ ) {
      val c = witness(existentialAxiom)
      have( ⊥ ) by Tautology.from(
            c.definition, emptySetAxiom of (x := c))
    }
  \end{lstlisting}
  \caption{An example use of local definitions in Lisa}
  \label{fig:localDefinitionExample}
  \end{figure}


\section{DSL}

\subsection{Instantiations with ``of''}

With lisa's kernel, it is possible to instantiate a theorem proving $P(x)$ to obtain a proof of $P(t)$, for any term $t$, using the \texttt{Inst} rule from \autoref{fig:deduct_rules_1}. Lisa's DSL provides a more convenient way to do so, using the \lstinline|of| keyword. It is used like so:
\begin{lstlisting}[language=lisa, frame=single]
  val ax = Axiom(P(x))
  val falso = Theorem(P(c) /\ P(d)) {
    have(thesis) by RightAnd(ax of (x := c), ax of (x := d))
  }
\end{lstlisting}
\lstinline|x := d| is called a substitution pair, and is equivalent to the tuple \lstinline|(x, d)|. Arbitrarily many substitution pairs can be given as argument to \lstinline|of|, and the instantiations are done simultaneously. \lstinline|ax of (x := c)| is called an \lstinline|InstantiatedFact|, whose statement is $P(c)$, and which can be used exactly like theorems, axioms and intermediate steps in the proof. Internally, Lisa produces a proof step corresponding to the instantiation using the \texttt{Inst} rule.

The \lstinline|of| keyword can also instantiate universally quantified formulas of a fact, when it contains a single formula. For example, the following code is valid:
\begin{lstlisting}[language=lisa, frame=single]
  val ax = Axiom(∀(x, P(x)))
  val thm = Theorem(P(c) /\ P(d)) {
    have(thesis) by RightAnd(ax of c, ax of d)
  }
\end{lstlisting}
Here, \lstinline|ax of c| is a fact whose proven statement is again $P(c)$. It is possible to instantiate multiple $\forall$ quantifiers at once. For example if \lstinline|ax| is an axiom of the form $\forall x, \forall y, P(x, y)$, then \lstinline|ax of (c, d)| is a fact whose proven statement is $P(c, d)$. It is also possible to combine instantiation of free symbols and quantified variables. For example, if \lstinline|ax| is an axiom of the form $\forall x, \forall y, P(x, y)$, then \lstinline|ax of (c, y, P := ===)| is a fact whose proven statement is $(c = y)$.

Formally, the \lstinline|of| keyword takes as argument arbitrarily many terms and substitution pairs. If there is at least one term given as argument, the base fact must have a single universally quantified formula on the right (an arbitrarily many formulas on the left). The number of given terms must be at most the number of leading universal quantifiers. Moreover, a substitution cannot instantiate any locked symbol (i.e. a symbol part of an assumption or definition). The ordering of substitution pairs does not matter, but the ordering of terms does. The resulting fact is obtained by first replacing the free symbols in the formula by the given substitution pairs, and then instantiating the quantified variables in the formula by the given terms

In general, for the following proof
\begin{lstlisting}[language=lisa, frame=single]
  val ax = Axiom(∀(x, ∀(y, P(x, y))))
  val thm = Theorem(c == d) {
    have(thesis) by Restate.from(ax of (c, d, P := ===))
  }
\end{lstlisting}
Lisa will produce the following inner statements:
\begin{lstlisting}
   -1 Import 0                 (  ) ⊢  ∀(x, ∀(y, P(x, y)))
    0 SequentInstantiationRule (  ) ⊢  c === d
\end{lstlisting}

